\begin{introduction}
This project is an experiment with, Ridge,lasso,Bagging, with DecisionTreeClassifier, and Adapost. Then i implemented a  
convolutional neural networks , which defined as a class of deep neural networks used in deep learning and machine learning.

Convolutional neural networks are usually used for visual imagery, helping the computer identify and learn from images. 
I will use deep learning methods to build neural networks, train the model, execute the algorithm and get the result in order to reveal how the deep neural network works.

Deep learning models are trained using large sets of labeled data and neural network architectures that learn features directly from the data. Networks will be trained on a specific data set.

I will focus on one of the most popular types of deep neural networks called Consensual Neural Network known as ConvNets.

\hfill \break
\hfill \break

A. Goals
1) Find a good dataset to be used as  datasets for the models.
2) Design the the models, and find fitting hyperparameters in research.
3) Train the models, with the dataset.
4) Evaluate the models and discuss the results.
5) Optimize the models

B. Significance
It is important to achieve the research goal, which is to compare the performance of different algorithms, and methods.

In this paper, I aim to how to train the model using a public data source to see which method will help predict age from facess with a good performance.


C. Paper Structure
The paper is structured as follows: (1) introduction, (2)
background, (3) model, (4) results, (5) discussion, and (6) conclusion. The background chapter provides
an overview of the most important concepts for this research
project including other relevant literature. The model chapter
details the topology, training- and evaluation method of the
models. 


\subsubsection{Related work}
\section{Related Work}
Through the initial literature review of work described in Abdolrashidi et al. (2020) [1],
Greco et al. (2020) [2], Haseena et al. (2022) [3], Liao et al. (2018) [4], Dhomne et al. (2018),
Nyeong-Ho et al. (2018) and Steven et al. (2019), we have decided on our topic and approach.

In Abdolrashidi et al. (2020), the authors proposed a two-headed attentional CNN
where one head is responsible for gender prediction while the other is for age prediction, and
the result of gender prediction is used to facilitate age prediction. An ensemble of a residual
attentional network (RAN) and residual convolutional network (ResNet) is also applied to
further boost the overall accuracy.
In Greco et al. (2020), the authors based their model on so-called MobileNets whose
architecture is particularly suited to balance the trade-off between accuracy and speed, such
that the final model can be applied on platforms with limited processing power such as
mobile devices.
In Haseena et al. (2022), the authors proposed pre-processing steps including image
processing, noise removal, face detection and alignment and applied a deep convolutional
neural network with 5 convolution layers and 2 fully connected layers for feature extraction.
Finally, classification was made with support vector machines. In Liao et al. (2018), a similar
approach is proposed for feature extraction. However, the authors proposed a
divide-and-rule method to reduce the complexity of age prediction by simplifying the problem
into comparing two faces and predicting which one is of an older age.
In Nyeong-Ho et al. (2018), the authors proposed a novel ordinal regression
algorithm, called moving window regression. The notion of relative rank (in this case both
age and gender, separately) which is a new order representation scheme for input and
reference instances was introduced. Then global and local relative regressors to predict
ranks within entire and specific rank ranges are developed. After that, there is a need to
refine an initial rank estimate iteratively by selecting two reference instances to form a search
window and then estimating the rank within the window. Extensive experiments results show
that the proposed algorithm achieves state-of-the-art performances on various benchmark
datasets for facial age estimation and historical color image classification.
In Steven et al. (2019), the authors introduced a continual learning approach to learn
new tasks without forgetting. Unlike previous methods growing monotonically in size, their
approach maintains the compactness in continual learning. The proposed packing and
expanding method is effective and easy to implement, which can iteratively shrink and
enlarge the model to integrate new functions. They claim that their integrated multitask model
can achieve similar accuracy with only 39.9% of the original size.

\subsubsection{Libraries}

\hfill \break
1.Keras: 

Keras is an open-source software library that provides a Python interface for artificial neural networks. Keras acts as an interface for the TensorFlow library[7].
\hfill \break
\hfill \break
2.TensorFlow : TensorFlow is a free and open-source software library for machine learning and artificial intelligence. It can be used across a range of tasks but has a particular focus on training and inference of deep neural networks[8].
\hfill \break
\hfill \break

\subsubsection{Datasets}
The data set used in this experiment is the UTK data set, which is a csv file that is publicly available. Data collection requires a lot of data cleaning and pre-processing. The dataset contains facial images that have been categorized based on age, gender, and ethnicity. The dataset includes 27,305 rows and 5 columns[9].



\end{introduction}